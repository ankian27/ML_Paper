\chapter{Implementation}
\label{chap:impl}

\section{First section}

You may need a nice figure, which you can algorithmically render using the Tikz package. You should really check out the Texample web site where several nice tikz examples are provided (http://www.texample.net/tikz/examples/all/).

\begin{figure}
\centering
\begin{tikzpicture}

\def \n {5}
\def \radius {3cm}
\def \margin {8} % margin in angles, depends on the radius

\foreach \s in {1,...,\n}
{
  \node[draw, circle] at ({360/\n * (\s - 1)}:\radius) {$\s$};
  \draw[->, >=latex] ({360/\n * (\s - 1)+\margin}:\radius) 
    arc ({360/\n * (\s - 1)+\margin}:{360/\n * (\s)-\margin}:\radius);
}
\end{tikzpicture}
\caption{Clear and concise figure captions are important to write. This one illustrates the cycle of a graph.}
\label{fig:tikzexample}
\end{figure}

\section{Initial Section}

\begin{itemize}
\item What language am I doing this in?
\item What is the question the urban p[lanner want solved?
\end{itemize}
 
\begin{enumerate}
\item What languge am I doing this in?
\item What is the question the urban p[lanner want solved?
\item new item
\begin{itemize}
\item new 1
\item new 2
\end{itemize}
\end{enumerate}

A graph rendered with the Tikz package is shown in \autoref{fig:tikzexample}.

\subsection{Subsection One}
\subsection{Subsection Two}
\subsection{Subsection Three}

\section{New Section For Next Important Topic}

\subsection{Algorithm Initialization}
\subsection{Atomic Operations}

You may even need code in your thesis. Here is a way to nicely include code with \LaTeX using the listings package.
{\singlespace
\begin{lstlisting}
for (unsigned int idx=0; idx<maxSize; idx++) {
  atomic_add( idx );
}
\end{lstlisting}
}

\subsection{Programming Style}
\subsubsection{Explaining Fine Detail Here}
\TODO{Make sure to finish this!}

\subsubsection{Last Subsection}
