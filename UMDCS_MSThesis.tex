% M.S. Computer Science Thesis
\documentclass{umdthesis}

\usepackage{setspace}

\usepackage[utf8]{inputenc}
\usepackage{array}
\usepackage{wrapfig}
\usepackage{multirow}
\usepackage{tabu}


% This package will nag about old-style LaTeX use.  Feel free to
% uncomment if you care about such issues.  \usepackage[l2tabu,
% orthodox]{nag}

% Note as of 11/18/14
% At the time of this update, the website:
% http://www.olivierverdier.com/posts/2013/07/15/modern-latex/ had
% useful information for Modern LaTeX... some of it is included here.

% Uses new biblatex system rather than traditional bibtex
\usepackage[backref=true, isbn=true, url=true, firstinits=true, maxnames=20, style=numeric, backend=biber]{biblatex}

% allows the Table of Contents to include the figure listings.
\usepackage{tocbibind}
% changes the name from Bibliography to References which is more appropriate for CS
\DefineBibliographyStrings{english}{%
  bibliography = {References},
}
%\renewcommand\bibname{References}

% Your references are placed into the .bib file and then specified here:
\addbibresource{UMDCS_MSThesis.bib}

% \usepackage{fontspec}
% \setmainfont{TimesNewRomanPSMT}
% \setmathfont{CambriaMath}
% \setmainfont{Times}
% \setsansfont{Helvetica}

% This will highlight code examples with nice coloring if you have that in your thesis.
% 
% This example uses C++, but there are many nice examples for other languages too.
%
% Nice C++ listings example derived from http://timmurphy.org/2014/01/27/displaying-code-in-latex-documents/
\definecolor{listinggray}{gray}{0.9}
\definecolor{lbcolor}{rgb}{0.95,0.95,0.95}

\lstset{
  backgroundcolor=\color{lbcolor},
  language=C++, % if you need to change the language...
  frame=lines, % draw a frame at the top and bottom of the code block
  tabsize=3, % tab space width
  captionpos=b,
  basicstyle=\footnotesize,
  breaklines=true,
  showstringspaces=false, % don't mark spaces in strings
  numbers=left, % display line numbers on the left
%  commentstyle=\color{green}, % comment color
  keywordstyle=\color{blue}, % keyword color
  stringstyle=\color{red} % string color
}

\RequirePackage{amsmath}
\RequirePackage{amssymb} 

% Supplied by Matt Overby - Spring 2014 - This is a nice command that
% lets you mark sections you will come back to later.
\newcommand{\TODO}[1]{\colorbox{yellow}{\textbf{TODO}: #1}}

\title{\textbf{\LARGE SMS Spam Filtering using WEKA}}
\author{\Large Ankit Anand Gupta}

\advisor{For CS 8751 - Advanced Machine Learning \vskip \li Dr. Richard Maclin }

%\professor{Dr. Richard Maclin}  

\copyrightyear{2017} 


\abstractfile{abstract}

\begin{document} 

	\frontmatter 
 
        % You could save some space when initially printing (if you
	% really need to) by uncommenting this line to single
	% space
        %
        % \singlespace
      
        % Ideally the Introduction to your thesis 
  %\doublespacing      
	\chapter{Introduction}
\label{chap:intro}

It's always good to introduce your (1) problem, (2) why it is interesting, (3) what you did, and (4) roughly, ssssssssssssssssssssssssssssssssssssssssssssssssssssssssssssss
how well did it work. You might even have citations in here, as in this paper~\cite{Asawa:2008:TDT}.



 

	 
        % Any background necessary to understand your thesis.  This 
        % can also contain the related work too. 
	\chapter{Background}
\label{chap:background}

\section{Background}

In this work, we investigate human computer interaction. 

\subsection{Robot Interfaces}

In seminar today, we looked for papers on the ACM Digital Library. The
following paper is about virtual reality \cite{Kreylos:2006:ESW:1128923.1128948}.

Just found another paper on robotics \TODO{READ Soon!} \cite{Drascic89}

\section{Previous Work}

In seminar today, we looked for papers on the ACM Digital Library. The
following paper is about virtual reality
\cite{Kreylos:2006:ESW:1128923.1128948}.

Found another paper... think the tile has VRGP in it... make sure to read.





 
 
        % Of course, if you wanted a separate chapter for Related
        % Work, you could make one too. 
	% \include{chap_relatedwork}     

        % The major content for your thesis!  What did you do and how   
        % did you do it!
	\chapter{Implementation}
\label{chap:impl}

\section{First section}

You may need a nice figure, which you can algorithmically render using the Tikz package. You should really check out the Texample web site where several nice tikz examples are provided (http://www.texample.net/tikz/examples/all/).

\begin{figure}
\centering
\begin{tikzpicture}

\def \n {5}
\def \radius {3cm}
\def \margin {8} % margin in angles, depends on the radius

\foreach \s in {1,...,\n}
{
  \node[draw, circle] at ({360/\n * (\s - 1)}:\radius) {$\s$};
  \draw[->, >=latex] ({360/\n * (\s - 1)+\margin}:\radius) 
    arc ({360/\n * (\s - 1)+\margin}:{360/\n * (\s)-\margin}:\radius);
}
\end{tikzpicture}
\caption{Clear and concise figure captions are important to write. This one illustrates the cycle of a graph.}
\label{fig:tikzexample}
\end{figure}

\section{Initial Section}

\begin{itemize}
\item What language am I doing this in?
\item What is the question the urban p[lanner want solved?
\end{itemize}
 
\begin{enumerate}
\item What languge am I doing this in?
\item What is the question the urban p[lanner want solved?
\item new item
\begin{itemize}
\item new 1
\item new 2
\end{itemize}
\end{enumerate}

A graph rendered with the Tikz package is shown in \autoref{fig:tikzexample}.

\subsection{Subsection One}
\subsection{Subsection Two}
\subsection{Subsection Three}

\section{New Section For Next Important Topic}

\subsection{Algorithm Initialization}
\subsection{Atomic Operations}

You may even need code in your thesis. Here is a way to nicely include code with \LaTeX using the listings package.
{\singlespace
\begin{lstlisting}
for (unsigned int idx=0; idx<maxSize; idx++) {
  atomic_add( idx );
}
\end{lstlisting}
}

\subsection{Programming Style}
\subsubsection{Explaining Fine Detail Here}

\TODO{Make sure to finish this!}

\subsubsection{Last Subsection}
  

        % It's always good to have a results chapter so you can 
        % present how well your ideas and implementations worked
	\chapter{Results}
\label{chap:results}

\vspace{1cm}

%---------------------------------------------------------------------
%Horizontal lines as row separators
\begin{center}
 \begin{tabular}{|c|c|c|c|c|} 
 \hline
 Algorithm & SC\% & BH\% & Acc\%  & MCC \\ [0.5ex] 
 \hline\hline
 Naive Bayes NB & 85.5\% & 1.2\% & 96.9\% & 0.869  \\ 
 \hline
 Naive Bayes Multinomial NB MN & 92.6\% & 1.0\% & 97.7\% & 0.902  \\ 
 \hline
 Boosted Naive Bayes BNB & 85.5\% & 1.2\% & 96.9\% & 0.869  \\ 
 \hline
 SVM & 81.6\% & 1.3\% & 96.3\% & 0.841  \\ 
 \hline
 SVM Stochastic Gradient Descent & 86\% & 0.6\% & 97.6\% & 0.896  \\ 
 \hline
 Voted Perceptron & 81.4\% & 1.6\% & 96\% & 0.829  \\ 
 \hline
 PART & 71.1\% & 3.1\% & 93.1\% & 0.710  \\ 
 \hline
 J48 & 58.7\% & 1.3\% & 92.9\% & 0.684  \\ 
 \hline
 1NN & 63.5\% & 23\% & 84.4\% & 0.309  \\ 
 \hline
 3NN & 43.9\% & 25.2\% & 80\% & 0.144 \\ [1ex] 
 \hline
 \end{tabular}
 
      \small
      \item Table 1 - The results when the the data is split 30\% and 70\% as 
      Training and Test.
    
 
\end{center}

%---------------------------------------------------------------------
\vspace{1cm}

%---------------------------------------------------------------------
%Horizontal lines as row separators
\begin{center}
 \begin{tabular}{|c|c|c|c|c|} 
 \hline
 Algorithm & SC\% & BH\% & Acc\%  & MCC \\ [0.5ex] 
 \hline\hline
 Naive Bayes NB & 86.6\% & 1.5\% & 96.9\% & 0.865  \\ 
 \hline
 Naive Bayes Multinomial NB MN & 92.6\% & 1.0\% & 98.1\% & 0.918  \\ 
 \hline
 Bossted Naive Bayes BNB & 86.6\% & 1.5\% & 96.9\% & 0.865  \\ 
 \hline
 SVM & 75.5\% & 0\% & 96.8\% & 0.918  \\ 
 \hline
 SVM Stochastic Gradient Descent & 89.4\% & 0.1\% & 97.8\% & 0.903  \\ 
 \hline
 Voted Perceptron & 87.7\% & 1\% & 97.5\% & 0.890  \\ 
 \hline
 PART & 85\% & 1.5\% & 96.7\% & 0.856  \\ 
 \hline
 J48 & 80.9\% & 0.9\% & 96.6\% & 0.849  \\ 
 \hline
 1NN & 66.3\% & 0.3\% & 95.3\% & 0.781  \\ 
 \hline
 3NN & 49.3\% & 0\% & 93.7\% & 0.676 \\ [1ex] 
 \hline
 \end{tabular}
 
      \small
      \item Table 1 - The results when the the data is 10-fold cross-validated.
    
 
\end{center}
%\begin{center}

The quick brown fox jumps right over the lazy dog and the starting of the treat the dunsfj fksths dijfji sdifjs dfjsid d sddfsdf dsgjngs 
kfsnfsdf dsf sdfsdfsd fsdf sdofkf sthe 
%\end{center}

 
%---------------------------------------------------------------------





        % Time to wrap up the thesis with a discussion of your ideas
        % and knowledge that you generated, along with any important
        % insights, or things you learned.  You can include ideas for
        % future extensions and effort here too.
	\chapter{Conclusions}
\label{chap:conclusions}

How can you wrap this up?


	
        % If you need an appendix (or appendices), they can be added here 
	\appendix
\chapter{Appendix A}
%\usepackage{setspace}
%\doublespacing

Do you need an Appendix?  You can include several of them if you want.

	 
	
        % And finally, don't forget the references and bibliography.
        % You can add entries to the file UMDCS_Thesis.bib for your
        % references.  You then need to ``cite'' them in the tex files
        % by using the ~\cite{ReferenceID} tags.
        %
        % and make sure to include the bib in the TOC
	\newpage
        \printbibliography[heading=bibintoc]
	
\end{document}
